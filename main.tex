\documentclass{article}

\usepackage{geometry}
\usepackage{hyperref}

\title {Khidmat: Fixit School}

\author{
  Ubaid Ali Faruqi\\ uf02900
  \and
  Aiman Ahmed Moin\\ am02886
  \and
  Syed Muhammad Hasan\\ sr02926
  \and
  Syeda Areeba Kazmi\\ sk02901
  \and 
  Ahsan Qadeer\\ aq02756
  \and
  Tasneem Adnan\\ ta02903
}
\date{}  

\begin{document}
\maketitle

% Use first person plural (we, us) even if you did the Khidmat individually.

% An introduction of the project, no more than 2 sentences. Provide the highest level of detail only. Other details will come later.
% Typically, "This project is to <short description of porject> for/at <client>."
This project is to initiate a program for the students of \textbf{Fixit} School that aims to develop and
enhance the computational soft skills of under-privileged  students.

% About the client.


% About the project.
An important and basic skill that all students must be equipped with is a strong command over the Microsoft Office utilities – whether they are looking for higher educational opportunities or jobs. As part of our workshop, we intend on working with Microsoft Word, Microsoft PowerPoint and Microsoft Excel amongst our students such that they may use these utilities in the most proficient manner and showcase these skills with freedom in class activities.

% About the plan of work.
We will conduct 4 hours session each day. For the 1st hour, we will be giving theoretical lecture. In the 2nd hour, we will be giving a demo on our computers, and for the remaining two hours the students will be having hands-on experience and practicing on computer by themselves.

% Copy-paste this section with necessary modifcations for each week.
\newpage % Start the report for each week on a new page.
\section*{Week 1: 11--17 July, 2018}

% A summary, maximum 2 sentences, of this week's activities.
We spent this week providing the students with the skills of Microsoft-Word, and Microsoft-Powerpoint.

\begin{tabular}{|l|l|l|l|}
  \hline
  Item 	& Activity & Time & ID \\\hline\hline
  1	& Group 1 hands-on session (including demo&&\\ &and practical test/revision by students) & 3.5 hrs & sk02901, ta02903 \\\hline
  2	& Group 2 hands-on session (including demo&&\\ &and practical test/revision by students) & 3.5 hrs & uf02900, am02886 \\\hline
  3	& Group 3 Theory session & 2 hrs & aq02756, sr02926 \\\hline
  4	& Group 3 hands-on session (a quick demo&&\\ &of the theory topics followed by test) & 1.5 hrs & aq02756, sr02926 \\\hline
  5	& De-Stress! & 0.5 hrs & All \\\hline    
\end{tabular}

The total time spent on the Khidmat this week is as follows.

\begin{tabular}{|l|l|}
  \hline
  ID & Total Hours\\\hline\hline
  uf02900 & 24 hours\\\hline
  am02886 & 24 hours\\\hline
  sr02926 & 24 hours\\\hline
  sk02901 & 24 hours\\\hline
  aq02756 & 24 hours\\\hline
  ta02903 & 24 hours\\\hline
\end{tabular}

% Other weeks ...
\section*{Week 2: 18--24 July, 2018}
% A summary, maximum 2 sentences, of this week's activities.
This week we continued with the remaining concepts of Microsoft-Powerpoint, and ended with the skills of Microsoft- Excel.

\begin{tabular}{|l|l|l|l|}
  \hline
  Item 	& Activity & Time & ID \\\hline\hline
  1	& Group 1 hands-on session (including demo&&\\ &and practical test/revision by students) & 3.5 hrs & sk02901, ta02903 \\\hline
  2	& Group 2 hands-on session (including demo&&\\ &and practical test/revision by students) & 3.5 hrs & uf02900, am02886 \\\hline
  3	& Group 3 Theory session & 2 hrs & aq02756, sr02926 \\\hline
  4	& Group 3 hands-on session (a quick demo&&\\ &of the theory topics followed by test) & 1.5 hrs & aq02756, sr02926 \\\hline
  5	& De-Stress! & 0.5 hrs & All \\\hline    
\end{tabular}

The total time spent on the Khidmat this week is as follows.

\begin{tabular}{|l|l|}
  \hline
  ID & Total Hours\\\hline\hline
  uf02900 & 24 hours\\\hline
  am02886 & 24 hours\\\hline
  sr02926 & 24 hours\\\hline
  sk02901 & 24 hours\\\hline
  aq02756 & 24 hours\\\hline
  ta02903 & 24 hours\\\hline
\end{tabular}
\newpage
\section*{Conclusion}

% Remind the reader about the project. Summarise your activities over the course of the project.
By the end of our project with the students of \textbf{Fixit} School, we were able to familiarize them with the three most important Microsoft Office utilities and Microsoft Paint. However, since students had little to no experience of using Computers, we had a very difficult task ahead of us, for e.g. It took a lot of time for students to understand the Keyboard Layout itself.\\
Even so, we were extremely pleased to see the development in a lot of students by the end of the workshop. The students were keen to learn and gain hands-on experience and we had managed to complete our revised Sylalbus - adjusted by the comfort level of each group - on time. 

\newpage
\thispagestyle{empty}
% Show your external supervisor your report, especially the weekly upates; have them sign a printed copy of this page; scan the signed page; and include the scanned page in this document as an image.

\begin{center}
  {\Large\bf Khidmat Completion Form}\\[5pt]
  \small To be completed by the external supervisor.  
\end{center}
\bigskip

\noindent{\it Please use the space below to provide any comments you may have on the students' performance, the Khidmat program, or any other feedback you want to share with Habib University's Khidmat committee. We can also be reached at \href{mailto:khidmat@sse.habib.edu.pk}{khidmat@sse.habib.edu.pk}.}
\vfill

\begin{center}
  \rule{.8\textwidth}{.5pt}
\end{center}
\medskip

% Insert your name below.

I hereby certify that I supervised UBAID ALI, AIMAN AHMED MOIN, SYED MUHAMMAD HASAN, AREEBA KAZMI, AHSAN QADEER, and TASNEEM ADNAN for the Khidmat described in this report. Furthermore, that I have read and agree with the weekly updates included in this report. My signature below marks the successful completion of the work undertaken for the Khidmat.\\
\bigskip
\bigskip

\noindent\begin{tabular}{@{}p{.6\textwidth}@{\hspace{.1\textwidth}}p{.3\textwidth}}
  \hrulefill &   \hrulefill \\
  Name and signature & Location and date
\end{tabular}

\end{document}
